
%\conclusions
%% \conclusions[modified heading if necessary]
\cite{Fontaine2011} suggests a ``bank model'' for SOM vs. litter degradation in soils. N-rich recalcitrant carbon is decomposed when N content is low, while N-fertilized soils principally degrade carbohydrate-rich and N-poor litter leachates. This leads to increased N mobilization in N-poor soils when additional (labile) carbon is available. On term of the microbial community, the production of oxidative enzymes is needed to degrade SOM, so investment in the production of these enzymes is up-regulated under C-rich and N-poor conditions. Our results suggest, that similar controls exist in litter decomposition.

%[?Lignin is not rejected for its intrinsic recalcitrance, but because it has little to offer to a community with rare labile carbon..?...]