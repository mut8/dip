
%\introduction
%% \introduction[modified heading if necessary]https://www.n3tw0rk.org/search.php?search_id=newposts

%Litter decomposition controls nutrient recycling and the release of assimilated carbon. Therefore it is considered a key process in terrestric ecosystems. While controls over decomposition rates -litter chemistry, climate and decomposer comunities - are known and generally accepted, the nature of the chemical transformations involved still remains a black box [lit]. \cite{Prescott2010} emphasises that the amount and chemisty of the humified biomass sequested to soils after litter decomposition might be have a higher impact on carbon cycling than decomposition speed.
%ev. leachates as input of labile carbon to soils?

%While for [two decades] the different compound's roles in litter decomposition were thought assigned [and fixed], recent studies reopen the discussion. Doubt is cast the differences in recalcitrance attributed to compound classes, which were considered established knowledge a few years ago. \citep{Marschner2008}

Plant litter biomass is dominated by macromolecular compounds. Together, lignin, carbohydrate and protein polymers make up xx\% of litter dry mass, while leach-able substances (``DOM'') in litter account for only xx \%. The conversion of insoluble compounds in particular organic matter (``POM'') into soluble substances is key process in litter decomposition: Microorganisms can only metabolize DOM directly, but rely on the excretion of extracellular enzymes to convert POM in DOM \citep{Klotzbucher2011,Bengtson2007,Marschner2003}.

Carbohydrates and protein polymerization is exactly controlled by catalytic enzymes. Lignin is result of a radical polymerization reaction of enzyme-activated hydroypropylphenol monomers. During lignin condensation other compounds - most prominently carbohydrates and protein - are incorporated into lignin structures \citep{Achyuthan2010}. Extracted (ADF) lignin fractions from fresh beech litter were found to have nitrogen contents twice as high as in bulk litter \citep{Dyckmans2002}. 

Conventional litter decomposition models [lit] follow the idea that macromolecules in litter form three independent carbon pools of increasing recalcitrance. These pools are attributed to (1) soluble compounds (most prominently starch), (2) cellulose and hemi-celluloses and (3) lignin. During decomposition, soluble compounds are easiest accessible for microbes and consumed first, followed by carbohydrates (i.e. cellulose). Lignin can be decomposed only by specialists and is not degraded until accumulated to a certain, critical level when it inhibits the degradation of other compounds \citep{Berg1980, Couteaux1995, Moorhead2006}.[more lit.] 

%[This concept was first described by , and stills forms the base of recent models and textbooks [lit!!hh]. 

One reason for the popularity of this model is that sizes of the three carbon pools can easily determined by proximate analysis. In these methods, litter cellulose, hemi-celluloses and lignin content are determined by sequential extractions with selective solvents. Especially for lignin determination, these methods (``Klason''- and ``ADF''-lignin) were repeatedly criticize as unspecific \citep{Hatfield2005}. When analyzed with alternative methods (NMR, CuO-oxidation, Pyrolysis-GC/MS), extracted lignin fractions contain many other than the proclaimed substances. (i.e. \cite{Preston1997}, [lit CuO], lit[Pyr]). However, while not helpful in tracking the fate of lignin in litter decomposition, these fractions - re-labeled ``acid un-hydrolyzable residues'' (AUR) - can be used as an indicator for the content of the most recalcitrant carbon compounds in litter \citep{Prescott2010}. For the lipids and plant waxes also found in AUR fraction, neither their accumulation/depletion during decomposition nor the effect of their concentration on decomposition processes is known [!check!,!lit!]

%[The question of reference material]

%While it is possible to determine carbohydrate and protein contents by enzymatic or chemical hydrolysis and quantification of monomers, due to the arbitrary structure of it's polymers, this approach can not be used for lignin determination.

%Especially lignin determination by this method has been citized extensively \citep{Hatfield2005}, as acid unsoluble residues (AUR, the fraction formally known as lignin) contain a number of hydrophobic compounds like cutin, surface waxes [fatty acids] and condesed tannins (lit). 

Recent studies using more specific methods to determine litter lignin content (CuO - oxidation, pyr-GC/MS, NMR) question the previously assumed intrinsic recalcitrance of lignin. Mean residence times for lignin in soils were calculated from both laboratory and outdoor incubation of litter/soil mixtures. Lignin residence times found were no longer than other carbon compounds or bulk SOM \citep{Thevenot2010a, Bol2009} [more lit?]. For litter, lignin decomposition rates were found not to increase from early to late decomposition stages \citep{Klotzbucher2011}. Based on these results, the authors propose a new model for lignin degradation: fastest lignin degradation in litter decay occurs during early litter decomposition; lignin decomposition during late decomposition is limited by (dissolved organic) carbon availability. 

Neither the traditional 3-pool models nor \cite{Klotzbucher2011} elaborate the effect of macro-nutrient (nitrogen and phosphorous) availability on lignin decomposition. Nitrogen fertilization experiments on litter and soils suggest that litter nitrogen content affects lignin degradation: N addition increases mass loss rates in low-lignin litter while slowing down decomposition in lignin-rich litter \citep{Knorr2005}. High nitrogen levels were reported to inhibit lignolytic enzyme in forest soils\citep{Sinsabaugh2010}. Cellulose addition lead to a higher mineralization of SOM in fertilized than in unfertilized soils \citep{Fontaine2011}. However, results of artificial fertilization can not be compared to different ``natural'' nutrient gradients. To our knowledge, no other experiment has yet compared effects of intra-specific variance in litter nutrient contents on decomposition processes. N-fertilization experiments can simulate increased N-deposition rates. To simulate variations litter C:N ratios, our approach is preferable, because potential changes in litter N content will most probably affect complex POM substrates. There, N location and accessibility is different of the low molecular weight N species available for fertilization experiments. 


% [Elevated N deposition and elevated soil N content increase litter N contents, while a recent meta-study hints that elevated atmospheric CO$_{2}$ concentrations cause wider litter C:N ratios \citep{Luo2006}. Therefore it is important to assess the impact of shifts in litter C:N ratio on decomposition processes and the chemical nature of the resulting organic matter to predict feedback mechanisms of anthropogenic alterations of global carbon and nitrogen cycles. While predictions of changes in mass loss rates under alternated litter C:N ratios are abound [lit.], no studies on changes in the quality of litter biomass during and after decomposition or of the dynamics of accumulation/depletion of fragments of the litter biomass during decomposition exist yet.]

%Also, interdependance of these pools seems much higher than expected. Lignified cellulose and protein is not accessable without degrading lignin... Ligin fractions extracted from (fresh) beech litter have beed demonstrated to have a very narrow C:N ratio if compared to bulk litter, nonligneous cell walls or soluble matter. It is assumed, that proteins are covalently bound to lignin polymers during polymerization. For litter decomposing microbia this implies that degrading lignin increases nitrogen availability, while degrading non-lignified carbohydrates yields more metabolic energy, but no additional nitrogen. In an incubation eperiment with a litter/soil mixture, N addition stimulated carbon mineralization during the first weeks of incubation but slowed it down duringThe after  with  ratios after climate chamber incubation for up to 15 month. late decomposition.

%In temperate forests, litter decomposition is generally considered nitrogen limited (lit).
%several experiments report retention times for lignin in soil [not much higher than other soil components].

Several recent studies apply analytical pyrolysis (Pyr-GC/MS) to characterize complex natural organic polymers like soil organic matter (SOM, \cite{Vancampenhout2010}[more lit here]). 
%Only a limited number of pyrolysis studies comparing different decomposition levels have been done. 
Pyrolysis-based decomposition studies usually focus on the woody material \citep{Vinciguerra2007} or  soil/litter mixtures [lit! - z.b. gleixner ca.1999]. Microbial decomposition of straw was followed by Pyr-GC/MS by [lit] We found only one study analyzing different stages of litter decomposition with analytical pyrolysis \citep{Franchini2002} and one recent study using a related technique (thermally assisted hydrolysis and methylation in \citep{Snajdr2011}. 

\cite{Kuder1998}

%European beech (\emph{Fagus silvaticus} L.) is the dominant forest building tree species in central europe. 
In this study we analyze samples of climate-chamber incubated beech litter varying in N and P content with Pyrolysis-GC/MS (pyr-GC/MS). The experiment was designed to study microbial decomposition, exclude decomposing fauna and keep climatic conditions constant. Extensive data on litter chemistry and and decomposition process rates are available for this samples from previously publications \citep{Mooshammer2011, Wanek2011, Leitner2011} as unpublished data [provided by the MicDiF national research network]. 

We focus on changes in lignin and carbohydrate content, assuming that

(1) Lignified biomass and non-lignified carbohydrates are alternatively degraded. Microorganisms have to allocate N in the production of different enzymes. When environmental conditions are constant, microbial substrate preference is determined by litter chemistry,

(2) While (non-lignified) carbohydrates are easier degraded than lignin and the resulting sugar monomers yield more energy, lignin degradation improves to accessibility of nitrogen (``lignin mining'', \cite{Craine2007}). 
 More lignin is decomposed when nitrogen availability is low, and high nitrogen availability inhibits lignin degradation.

(3) Lignin degradation is inhibited when little DOC is available and decomposition is energy limited (as proposed by \cite{Klotzbucher2011}). 

%Their accumulation and/or depletion during decomposition was extensively studied, nevertheless recent studies based on new methods to measure lignin content challange traditional models.

% Decomposition conditions control whether assimilated carbon is released immideately or sequestered to soil organic matter. Decomposition processes also control the quality of the resulting organic matter and therefore soil carbon recalcitrance. 

%\cite{Prescott2010} suggests the amount of remaining biomass more important than the decomposition speed. We suggest, the chemical nature of this remaining biomass might be a factor too. we should study the influences of environmental parametres on the chemical conposition of the remaining biomass. 

% but using methods based on detergent extraction no regarded unspecific. 

%Large scale experiments (lit!) conducted during the last years collected detailed data on chemical and ecological controls predict dry mass loss during litter decomposition. Nevertheless, understanding of chamical transformations in litter loss is still limited, as most studies limit their analysis of high molecular weight compounds to detergent extraction based methods (Klason Lignin, ... - lit). 


%\cite{Snajdr2011a} used THM (a pyrolysis-related method) in a litterbag expiriment, but limits analysis to determining lignin:carbohydrate ratios. 

%These studies often explain biomarkers found in SOM by their occurance in a plant polymers and propose a decendence from thos biomarkers \citep{Schellekens2009b}, some of them also analyze local vegetation for comparison.


% We follow three research questions:

%(1) To which does HMW carbon at different sites differ? 

%(2) Which substance classes are accumulaten/depleted during decomposition and is the rate of accumulation/depletion influences by nutrient compositoin? Nutrient addition studies show a decrease of phenole oxidizing enzymes (lit!), but those have not been shown in pyrolysis data yet.

%(3) Does HMW chemistry excercice control over decomposition processes? Do changes in HMW chemistry (i.e. accumulation of more recalcitrant substances) explain the decline of decomposition rates?




% Nitrogen availability has been shown to be rate limiting in beech litter (lit) ??\citep{Mooshammer2011}??.

%With increasing anthropogenic nitrogen deposition and contradicting studies on whether litter C:N ratio is change due to increasing anthmospheric CO$_2$ ratios, knowledge about effects of nutrient availability on the chemical properties of soil organic matter become increasingly relevant (lit.).

% Most experiments study nutrient control by fertilization. 
%Nutrient availability Litter stoichiometry...
%prescott -> metastudy zu litter CN aus den 90ern
%die aktuelle metastudy
%prescott N deposition 
%N-addition studys. few studies on stoichiometry in naturally variing systems.

% CN ratio changes >> 

