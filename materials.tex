\section{Material and methods}

\subsection{Litter decomposition experiment}
A detailed description of our litter decomposition experiment was published in \cite{Wanek2011}. Briefly, beech litter from four sites in Austria (Achenkirch (AK), Klausenleopoldsdorf(KL), Ossiach(OS), and Schottenwald(SW); refered to as litter types) was collected in October 2008. Litter was homogenized, sterilized and inoculated (1.5\% w/w) with a mixture of litter and soil to assure all litter types share the same initial microbial community. From each type, four samples of  litter were taken after inoculation.
Samples of 60g litter (fresh weight) were kept at 15 \textdegree C and 60\% water content in mesocosms for a duration between 2 weeks to 15 month. For each litter type 5 replicas were removed and analyzed after 14, 97, 181 and 375 days. 

\subsection{Bulk litter, extractable, and microbial biomass nutrient content}
To calculate litter mass loss, litter dry mass content was measurement in 5 g litter (fresh weight) after 48 h at 80 \textdegree C. Dried litter was ball-milled for further chemical analysis. Litter C and N content were determined using an elemental analyzer (Leco CN2000, Leco Corp., St. Joseph, MI, USA). Litter phosphorus content was measured  with ICP-AES (Vista-Pro, Varian, Darmstadt, Germany) after acid digestion as described by \cite{Henschler1988}.

To determine soluble C, N, and P contents, 1.8g litter (fresh weight) were extracted with 50 ml 0.5M K\textsubscript{2}SO\textsubscript{4}. Samples were shaken on a reciprocal shaker with the extractant for 30 minutes, filtered with ash-free filters and frozen at -20 \textdegree C until analysis. For quantifying microbial biomass C, N and P pools, the same extraction was used after chloroform fumigation . Microbial biomass was determined as the difference between fumigated and non-fumigated extractions. C and N concentration in extracts were determined with a TOC/TN analyzer (TOC-VCPH and TNM, Schimadzu), Phosphorous was determined photometrically [..] [lit: schinner 1996]

Substrate-consumer stoichiometric inbalances \emph{X:Y$_{inbal}$} were calculated as

 \begin{equation}
X:Y_{inbal}=\frac{X:Y_{litter}}{X:Y_{microbial}} \label{eq:resp.acc}
 \end{equation}

where \emph{X} and \emph{Y} stand for on of the elements C, N, or P.

\subsection{Microbial Respiration}
Respiration was monitored weekly in the mesocosms that were harvested after 181 month and for all mesocosms one days before the harvestg an infrared gas analyzer (IRGA, EGM4 with SRC1, PPSystems, USA). CO2 concentration was measured over 70 seconds and increase per second was calculated based on g dry weight of the litter. Measurements of ambient air were performed before and after each measurement to assess possible leaks or base-line drifts IRGA. %Measurements were conducted using the following settings: volume of the chamber 1551 cm³, area of the chamber 115 cm², linear measurement, at 15\textdegree.
Accumulated respiration after 181 month was calculated assuming linear transition between measurements, accumulated respiration after 475 days was estimated from respiration rates after 181 and 475 days. We do not base interpretation exclusively on the later estimate but use is mainly ..
%as stated in equation \ref{eq:resp.acc}, where \emph{Resp\textsubscript{acc}} stands for the accumulated respiration, \emph{Resp\textsubscript{n}} and \emph{t\textsubscript{n}} for the actual respiration at and the decomposition time until time point \emph{n}.

% \begin{equation}
% Resp_{acc}(t_{n})=\sum_{i=0}^n (Resp(t_{i})+Resp(t_{i+1}))*(t_{i+1}-t_{i})/2
% \label{eq:resp.acc}
% \end{equation}

\subsection{Enzyme activities}

Measurements of potential exo-enzyme activities for cellulases, peroxidases and phenoloxidase were described by \cite{Leitner2011}. Activities were determined with a series of micro-plate assays based on the hydrolysis of 4-methyl-$\beta$-D-cellobioside (cellulase) and L-3,4-di\-hydroxy\-phenyl\-alanin (oxidative enzymes). Products of enzyme catalyzed reactions were detected photometrically (oxidative enzymes) or flourometrically (cellulase). The method was was initially published by \cite{Marx2001} and \cite{Sinsabaugh1999}, we applied a modified variant as described in \cite{Kaiser2010b}. Enzyme activity was measured after 14, 87 and 181 days. In this study we use the quotient between cellulase and oxidative enzymes to describe litter microorganisms investments in the trade-off lignin and cellulose degradation. 

\subsection{Glucan depolymerization and carbon use efficiency}
Glucan depolymerization and Glucose consumption were measured by \cite{Leitner2011} after 14, 97, and 181 days using a new 13C pooldillution assay . Method and results were reported in \cite{Leitner2011}. Based on the their results, we calculated the carbon use efficiency \emph{CUE} based on g carbon metabolized:

\begin{equation}
CUE = (\frac{Glucose Consumption - Respiration}{Respiration})
\end{equation}

\subsection{Metaproteome (and Metatranscriptome?) analysis}
...

\subsection{Pyrolysis-GC/MS}
Pyrolysis-GC/MS was performed on a Pyroprobe 5250 pyrolysis system (CDS Analytical) coupled to a Thermo Trace gas chromatograph and a DSQ II MS detector (both Thermo Scientific) equipped with a carbowax colomn (Supelcowax 10, Sigma-Aldrich).

Litter analyzed was sampled immediately after inoculation and after 97, 181, and 375 days. 2-300 \textmu g dried and finely ball-milled litter were heated to 600\textdegree C for 10 seconds in helium atmosphere. The temperature of the valve oven and the transfer line to the GC injection port were set to 250\textdegree C,a 10x split injection was applied with the injector heated to 240\textdegree C. GC Oven temperature was constant at 50 \textdegree C for 2 minutes, followed by an increase of 7\textdegree C/min to a final temperature of 260 \textdegree C, which was held for 15 minutes. The transfer line was heated to 270 \textdegree C. The MS detector was set for electron ionization at 70 EV, the ion source was heated to 270\textdegree C. Detection was set to cycle between m/z 20 and 300 with a cycle time of 0.3 seconds.

Peaks were assignment was based on NiSt 05 MS library and comparison with reference material measured. 133 peaks were identified and selected for integration due to their hight abundance or diagnostic value. For each peak between one and four dominant mass fragments selected for high abundance and specificity were integrated (as done by i.e. \cite{Schellekens2009}). Peak areas are stated as \% of the sum of all integrated peaks of a sample. 

Pyrolysis products were assigned to their substances of origin by comparison to reference material, structural similarity and in accordance with literature (\cite{Ralph1991a, Schellekens2009,Chiavari1992}[more lit!]). The sum of all peak areas of the pyrolysis products of a class was calculated based on total ion current (TIC) peak areas. TIC peak areas are (1) less specific as areas of specific MS fragments and (2) integration was not possible for all peaks a/o all samples. Therefore a MS response factor Rf was calculated for each detected substance:

\begin{equation}
Rf = median (\frac{TIC peak area}{specific MS fragment peak area})
\end{equation}

Peak areas were multiplied by Rf before addition to calculate percentages of TIC area without loosing the specifity of integrating single m/z traces.

% However, during interpretation, we found little difference between direct sums of the integrated fragments and sums of corrected areas, indicating little sensitivity for exact 

Relative peak areas in both integrations are different from weight\%, but allow tracing of accumulation/depletion of these substance classes during decomposition \citep{Schellekens2009}. 

%check pyridol peak!!

%All nitrogen containing compounds and their sum were correlated to litter N content (R\textgreater0.49 for all  and R\textgreater0.79 for 8 of 11 pyrolysis products and the sum of all N containing compounds. For all correlations, p=***). all peaks were correlated to their sum with .... The cTIC sum correlated to PCA1 with R\textgreater0.8x (p=***=).

%Of the 28 lignin derived molecules were detected, 26 were positivly correlated to their sum (p=** or ***), 18 with R\textgreater0.07 (all p=***). Another the 10 non-lignin phenolic compounds found, of which 9 were positively correlated to the sum of all Phenoles, 7 with R\textgreater0.7.

\subsection{Statistical analysis}
All statistical analyses were performed with the software and statistical computing environment R using the R package ``vegan'' \citep{Oksanen2011}. If not mentioned otherwise, results were considered significant, when p\textless 0.05. All correlations refer to Pearson correlations.
All data presented was tested for significant differences between harvests and litter types. Normal distribution assumed but could not be tested due to the small number of cases per treatment (n=4-5). A substantial part of variables had heterogeneous variances when tested ẃith Levene's test. Therefore, (one-way) Welch anova was used to calculate significant differences between harvests within each litter type and litter types within each harvest (alpha=0.05). For post-hoc group assignment, paired Welch's t-tests with Bonferroni corrected p limits were used. Principal component analysis was performed using vegan function ``rda'' scaling variables.
